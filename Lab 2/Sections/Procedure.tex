\section{Procedure}
% Requirements
% State the procedure followed in taking the data. The procedure should be organized 
% in the most concise, logical order; it does NOT need to be chronological. Use past 
% tense to tell the reader how you made the measurements. The reason for making each 
% measurement should also be given if necessary. e.g. "Stagnation pressure was measured 
% at 12 locations across the duct in order to determine average velocity". Include the make 
% and model number for important equipment used in the study. For MecE301 reports 
% this section is usually one or two paragraphs and one or two schematic figures. 
% Presenting, and referring to, a schematic diagram of the set-up saves words (and time) 
% and helps reader comprehension

A schematic of the experimental setup is shown in Figure [insert figure]
The precis provided by the MEC E 301 course was followed \cite{lab2precis}. All measurements were taken from the serial monitor
of the Arduino IDE. During calibration, different circuit components and reference voltages were used to measure the voltage output of 
the PCB. 



% \forceindent The following procedure was followed as outlined in the lab precis \cite{lab2precis}. First, the Arduino Uno was 
% connected to the computer and the Arduino IDE was opened. The Arduino IDE was used to upload the \texttt{AnalogReadSerial} after 
% modifying the code \texttt{float voltage = sensorValue * (5.0 / 1024.0);} to correct the decimal-to-number conversion. 
% An additonal line was added \texttt{Serial.println(voltage, 3)} to print the voltage to the serial monitor with 3 decimal places. 

% First, measuring voltages with a 5V reference voltage was performed. The \texttt{5V} and \texttt{GND} pins on the Arduino Uno were 
% connected to the \texttt{5V\_VIN} and \texttt{GND} pins on the PCB. The \texttt{A0} pin on the Arduino Uno was connected to the output pin,
% \texttt{2.500V}. After uploading the sketch, ten values were recorded from the serial monitor. The \texttt{2.500V} output pin was then
% swapped to \texttt{1.800V}, \texttt{1.024V}, and \texttt{0.102V} and ten values were recorded for each. 

% Next, measuring voltages with a 3.3V reference voltage was performed. The AREF jumper on the MEC E 301 Shield was inserted, connecting the 
% 3.3V reference voltage to the AREF pin on the Arduino Uno. The code was modified to reflect the new reference voltage, \texttt{float voltage = sensorValue * (3.3 / 1024.0);}.
% The voltages \texttt{2.500}, \texttt{1.800}, \texttt{1.024}, and \texttt{0.102} were measured again and ten values were recorded for each.

% % next was using voltage divider to measure voltages
% Next, measuring voltages with a voltage divider was performed. The output pin on the PCB was connected to the \texttt{D10\_I} on the MEC E 301 Shield.
% The \texttt{D10\_O} pin on the MEC E 301 Shield was connected to the \texttt{A0} pin on the Arduino Uno. The code was modified to reflect the new range 
% of voltages, \texttt{float voltage = sensorValue * (33.0 / 1024.0);}. The voltages \texttt{2.500}, \texttt{1.800}, \texttt{1.024}, and \texttt{0.102} 
% were measured again and ten values were recorded for each.

% % using -10 to 10 v range
% Next, measuring voltages with a -10V to 10V range was performed. The \texttt{+/-10\_I} was connected to the PCB output pin. The \texttt{+/-10\_O} pin on the MEC E 301 Shield
% was connected to the \texttt{A0} pin on the Arduino Uno. The code was modified to reflect the new range of voltages, \texttt{float voltage = sensorValue * (20.0 / 1024.0) - 10.0;}.
% The voltages \texttt{2.500}, \texttt{1.800}, \texttt{1.024}, and \texttt{0.102} were measured again and ten values were recorded for each.

% %using amplifier
% Voltage measurements with an amplifier was performed. The \texttt{X10\_I} pin on the MEC E 301 Shield was connected to the PCB output pin. The \texttt{X10\_O} pin on the MEC E 301 Shield
% was connected to the \texttt{A0} pin on the Arduino Uno. The code was modified to reflect the new range of voltages, \texttt{float voltage = sensorValue * (0.33 / 1024.0);}.
% The voltages \texttt{2.500}, \texttt{1.800}, \texttt{1.024}, and \texttt{0.102} were measured again and ten values were recorded for each.

% % pcb time varying voltage
% Time varying voltage measurements were performed. The \texttt{SINE\_OP} pin on the PCB was connected to the \texttt{A0} pin on the Arduino Uno. The code was modified to reflect the new range of voltages, 
% \texttt{float voltage = sensorValue * (3.3 / 1024.0);}. The code was modified to ouptut a timestamp and voltage value and the baud rate was set to 115200. The serial monitor was opened and the
% voltage was recorded for 255 values.


