\section{}
A displacement field in a body is given by
\begin{align*}
    u = c(x^2 + 10) \\
    v = 2cyz \\
    w = c(-xy+z^2)
\end{align*}

where $c= 10^{-4}$. Determine the state of strain on an element position at (0, 2, 1). 

Calculate all 6 unique entries of the strain tensor:
\begin{align*}
    \epsilon_{x}|_{(0,2,1)} &= \frac{\partial u}{\partial x} = 2cx = 2(10^{-4})(0) = 0 \\ 
    \epsilon_{y}|_{(0,2,1)} &= \frac{\partial v}{\partial y} = 2cz = 2(10^{-4})(1) = 2(10^{-4}) \\
    \epsilon_{z}|_{(0,2,1)} &= \frac{\partial w}{\partial z} = 2cz = 2(10^{-4})(1) = 2(10^{-4}) \\
    \epsilon_{xy}|_{(0,2,1)} &= \frac{1}{2}\left(\frac{\partial u}{\partial y} + \frac{\partial v}{\partial x}\right) = \frac{1}{2}(0 + 0) = 0 \\
    \epsilon_{xz}|_{(0,2,1)} &= \frac{1}{2}\left(\frac{\partial u}{\partial z} + \frac{\partial w}{\partial x}\right) = \frac{1}{2}(0 + -cy) 
    = \frac{1}{2}(0 + -2(10^{-4}) = -10^{-4} \\
    \epsilon_{yz}|_{(0,2,1)} &= \frac{1}{2}\left(\frac{\partial v}{\partial z} + \frac{\partial w}{\partial y}\right) = \frac{1}{2}(2cy + -cx)
    = \frac{1}{2}(2(10^{-4})(2) + 0) = 2(10^{-4})
\end{align*}

Therefore, the strain tensor is
\begin{empheq}[box=\widefbox]{align*}
    % factor out 10^-4
    \epsilon = \begin{bmatrix}
        0 & 0 & -1 \\
        0 & 2 & 2 \\
        -1 & 2 & 2
    \end{bmatrix} \times 10^{-4}
\end{empheq}

