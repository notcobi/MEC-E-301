\section{Results and Discussion}
% Make a table summarizing the range, resolution, repeatability, accuracy 
% (determined from your experiments), and the manufacturer’s stated accuracy for 
% the five ways we configured the Arduino to measure analog voltages. (State the 
% range in V and all other values in mV). The resolution can be calculated using 
% Equation 5 of the lecture notes. The manufacturer claims that the accuracy of the 
% Uno’s A/D converter is +/– 2 least significant bits (in other words, +/– 2 times the 
% resolution if the voltage range is scaled by a reference voltage, voltage divider, 
% amplified etc). 
% Discuss the following:
% b) Discuss the trade-off between the resolution of the ADC and its range. 
% c) How does the repeatability compare to the accuracy?
% d) Is the accuracy of the Arduino improved using an external reference voltage of 3.3 
% V compared to the default 5 V reference voltage?
% e) The voltage sources used to calibrate the Uno have an accuracy of 0.05%. Is this 
% accuracy sufficient? Why or why not?
% f) Compare the accuracy you measured to the accuracy claimed by the manufacturer. 
% g) The voltage divider, voltage scaler circuit, and amplifier create additional error 
% when they are used with the Arduino. Assume they have an error of 1% of reading. 
% What error in mV do they cause when measuring a voltage at 0.102 V and at 2.500 
% V? Is it a significant source of error compared to the accuracy of the Arduino?

\subsection{Calibration of the Arduino Uno Results}
\begin{table}[ht]
    \caption{Range, Resolution, Repeatability, Accuracy, and Manufacturer's Accuracy for Various Ranges of the Arduino Uno}
    \label{tab:arduino-accuracy}
    \centering
    \small
    \begin{tabular}{lccccc}
        \toprule
        Arduino Config. & Range & Resolution & Repeatability & Acc. & Manuf. Acc. \\
        & (V) & (mV/LSB) & (mV) & (mV) & (mV) \\
        \midrule
        5V Ref. & 0.000 - 5.000 & 4.883 & 44.00 & 54.00 & 9.766 \\
        3.3V Ref. & 0.000 - 3.300 & 3.223 & 4.000 & 17.00 & 6.445 \\
        3.3V Ref., 10x VDiv & 0.00 - 33.00 & 32.23 & 32.00 & 83.00 & 64.45 \\
        3.3V Ref., [-10, 10]V & -10.00 - 10.00 & 19.53 & 0.000 & 24.00 & 39.06 \\
        3.3V Ref., 10x Amp. & 0.000 - 0.330 & 0.3223 & 0.000 & 0.000 & 0.6445 \\
        \bottomrule
    \end{tabular}
\end{table}
\FloatBarrier

\subsection{Calibration of the Arduino Uno Discussion}
\subsubsection{Trade-off Between Resolution and Range}
The trade-off between resolution and range is that as the range increases, the resolution decreases. This is because the number of bits
available to represent the range is fixed. For example, the Arduino Uno has 10 bits to represent the range of voltages. This means that the
Arduino Uno can represent $2^{10} = 1024$ different voltages. 

From Appendix \ref{sec:appendix-resolution}, if the range is multiplied by a factor of $k$, then the resolution is also multiplied by a factor of $k$, increasing resolution error.

\subsubsection{Repeatability Compared to Accuracy}
Repeatability is the maximum deviation between two measurements of the same reference quantity. Accuracy is the maximum deviation between the
measured value and the true value. Repeatability is a measure of precision while accuracy is a measure of correctness. Both measure different
aspects of the measurement system.

\subsubsection{Accuracy of Arduino Uno with 3.3V Reference Voltage}
% d) Is the accuracy of the Arduino improved using an external reference voltage of 3.3 
% V compared to the default 5 V reference voltage?

% from lab manual:
% As you probably noticed the accuracy of the default Ardiuno is not very good. This is 
% because the Arduino uses the voltage supplied over the USB cable by your computer as the 
% reference voltage for the analog-to-digital converter. The Arduino assumes this voltage is 
% 5 V but in reality, it can vary between 4.5 to 5.5 V depending on the computer. This can 
% result in large errors. To make much more accurate measurements a reference voltage can 
% be used with the Arduino’s A/D converter by suppling a known voltage to the AREF pin 
% on the Arduino. The MecE 301 shield contains a 3.300 V reference voltage (top left corner 
% of the shield) that can be supplied to the Arduino A/D converter. If a reference voltage is 
% used then the range of the A/D converter will be 0 V to 3.3 V. Thus, changing the reference 
% voltage also changes the range and resolution of the A/D. 

The accuracy for the 5V reference voltage is 54.00mV while the accuracy for the 3.3V reference voltage is 17.00mV. The accuracy of the Arduino Uno is improved
using the 3.3V reference voltage. Many computers do not supply exactly 5V over the USB cable and the Arduino's A/D converter assumes the voltage is 5V. 
Supplying a known voltage to the AREF pin from the MEC E 301 Shield allows for more accurate measurements.

\subsubsection{Accuracy of Voltage Sources}
% e) The voltage sources used to calibrate the Uno have an accuracy of 0.05%. Is this 
% accuracy sufficient? Why or why not?
From Appendix \ref{sec:appendix-voltage-source-accuracy}, if we use a conservative estimate for the accuracy of the voltage source, which occurs at 2.500V, then the accuracy is 1.250mV. ANSI/ISA 51.1 states that the accuracy of the standard
can be ignored should the accuracy be one tenth of the instrument tested. % probably should cite this

Comparing this to the accuracy of the Arduino Uno, in Table \ref{tab:arduino-accuracy}, the accuracy of 
the voltage source is sufficient for all reference values except the setup with the 3.3V reference voltage and 10x amplifier, which has an experimental accuracy of 0.000mV.


\subsubsection{Accuracy of Arduino Uno Compared to Manufacturer's Accuracy}
% f) Compare the accuracy you measured to the accuracy claimed by the manufacturer.
Referring to Appendix \ref{sec:appendix-arduino-accuracy}, the experimental accuracy was less than the manufacturer's accuracy for all configurations 
except the 3.3V reference voltage with a 10x amplifier and the 3.3V reference voltage with a [-10, 10]V range.

The 5V reference voltage has an accuracy of 54.00mV while the manufacturer's accuracy is 9.766mV. This may be attributed to the variable reference voltage supplied by the computer


\subsubsection{Additional Error from Voltage Divider, Voltage Scaler, and Amplifier}
% g) The voltage divider, voltage scaler circuit, and amplifier create additional error 
% when they are used with the Arduino. Assume they have an error of 1% of reading. 
% What error in mV do they cause when measuring a voltage at 0.102 V and at 2.500 
% V? Is it a significant source of error compared to the accuracy of the Arduino?

The calculation is shown in Appendix \ref{sec:appendix-circuit-component-accuracy}. The error at 0.102V is 1.020mV and the error at 2.500V is 25.00mV. The error at 0.102 
is not significant while the error at 2.500V is significant by ANSI/ISA 51.1.

\subsection{Time Varying Voltage Discussion}
% 5) Considering the calculations you made in part 4), discuss the following:
% a) How do the Arduino measurements of mean voltage, frequency, and peak-to-peak 
% voltage compare to the oscilloscope measurements?
% b) Is the precision uncertainty in the frequency measurements with the 10-bit or 5-bit 
% Arduino large? How could the measurement of frequency with the Arduino be 
% improved?
% c) Did the 10-bit or 5-bit Arduino measurement of peak-to-peak voltage have the 
% highest total uncertainty? Was the highest uncertainty caused by precision
% uncertainty or bias uncertainty? What recommendations would you make to
% improve the measurement of the peak-to-peak voltage with the Arduino?











        