\section{Appendix: Derivation of Resolution}
\label{sec:appendix-resolution}
\noindent The trade-off between resolution and range is that as the range increases, the resolution decreases. This is because the number of bits
available to represent the range is fixed. For example, the Arduino Uno has 10 bits to represent the range of voltages. This means that the
Arduino Uno can represent $2^{10} = 1024$ different voltages. If the range is 0 to 5V, then the resolution is 4.883mV/LSB. If the range is
0 to 10V, then the resolution is 9.766mV/LSB. 

\noindent This can be shown mathematically. Then the resolution is given by:
\[
\begin{aligned}
    \text{Resolution} &= \frac{V_{\text{ru}} - V_{\text{rl}}}{2^n} \\
\end{aligned}
\]
Let us assume the upper range $V_{\text{ru}}$ and lower range $V_{\text{rl}}$ are both multiplied by a factor of $k$.
\[
\begin{aligned}
    \text{Resolution'} &= \frac{kV_{\text{ru}} - kV_{\text{rl}}}{2^n} \\
    &= \frac{k(V_{\text{ru}} - V_{\text{rl}})}{2^n} \\
    &= k \text{Resolution} \\
\end{aligned}
\]