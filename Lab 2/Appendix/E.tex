\section{Time-Varying Voltage Measurements}
\label{sec:appendix-time-varying-voltage-measurements}

\noindent Below in Table \ref{tab:time-varying-voltage-mean-measurements} are the nominal measurements for the time-varying voltage for the 10-bit and 5-bit ADCs.
\begin{table}[h]
    \centering
    \caption{Nominal Measurements of Time-Varying Voltage}
    \label{tab:time-varying-voltage-mean-measurements}
    \begin{tabular}{lccc}
      \toprule
           & \multicolumn{3}{c}{Mean Measurements of Time-Varying Voltage} \\
    \cmidrule{2-4}
           & Frequency & Peak to Peak & Mean Voltage       \\
           & (Hz)      & (V)          & (V)           \\
    \midrule
    10-bit & 47.266  & 1.217        & 1.135         \\
    5-bit  & 47.372  & 1.2194       & 1.144         \\
    Oscilloscope & 48.54 & 1.28 & 1.17 \\
   \bottomrule
    \end{tabular}
\end{table}
\subsection{Mean Sample Calculation}
\noindent The mean voltage for the 10-bit and 5-bit ADCs was calculated using Excel using the \texttt{=AVERAGE()} function across 5 cycles, which was
approximately 60 data points. 

\subsection{Time-Varying Voltage Period Measurements}

\noindent Below is a table of the time-varying period measurements for the 10-bit and 5-bit ADCs. 
\begin{table}[h]
    \centering
    \caption{Time-Varying Period Measurements for 10-bit and 5-bit ADCs}
    \begin{tabular}{lcccccc}
    \toprule
       & \multicolumn{5}{c}{Period ($\mu$s)}    \\
    \cmidrule{2-6}
       & Cycle 1    & Cycle 2   & Cycle 3   & Cycle 4   & Cycle 5    \\
    \midrule
    10-bit & 21000 & 21132 & 21132 & 21124 & 21400  \\
    5-bit  & 20912 & 21104 & 21136 & 21096 & 21304  \\
    \bottomrule
    \end{tabular}
\end{table}

\noindent \textbf{Obtaining the nominal value (mean)}, standard deviation, and T-distribution inverse was done through Excel. The mean was calculated using the \texttt{=AVERAGE()} function 
across 5 cycles, which was approximately 60 data points. The standard deviation was calculated using the \texttt{=STDEV.S()} function. 
The T-distribution inverse was calculated using the \texttt{=T.INV()} function, where $\alpha = 0.05$ and $n = 5$. The results are shown in Table 
\ref{tab:time-varying-period-uncertainty-measurements}.

\begin{table}[h]
   \centering
   \caption{Time-Varying Period Uncertainty Measurements for 10-bit and 5-bit ADCs}
   \label{tab:time-varying-period-uncertainty-measurements}
   \begin{tabular}{lcccccc}
      \toprule
      & Nominal Value & STDEV & T-Inv & $P_x$ \\
      & ($\mu$s)      & ($\mu$s)           &                        & ($\mu$s)    \\
      \midrule
      10-bit & 21158 & 146.66 & 2.7764 & 182.10 \\
      5-bit  & 21110 & 139.42 & 2.7764 & 173.11 \\
      \bottomrule
   \end{tabular}
\end{table}

Sample calculations for the \textbf{10-bit} ADC are shown. To calculate the random uncertainty, the following equation was used:
\[
\begin{aligned}
   P_x &= t_{\alpha/2, n-1} \frac{\sigma}{\sqrt{n}} \\
         &= 2.7764 \frac{\qty{146.66}{\micro\second}}{\sqrt{5}} \\
         &= \boxed{\qty{182.10}{\micro\second}}
\end{aligned}
\]

\subsection{Time-Varying Voltage Frequency Calculations}
%\noindent Below is a table of the time-varying frequency measurements for the 10-bit and 5-bit ADCs.
% Frequency	
% nominal value	Uncertainty
% 10-Bit	47.264	0.40680
% 5-Bit	47.370	0.38844
\begin{table}[h]
   \centering
   \caption{Time-Varying Frequency Results for 10-bit and 5-bit ADCs}
   \label{tab:time-varying-frequency-measurements}
   \begin{tabular}{lcc}
   \toprule
      & \multicolumn{2}{c}{Frequency (Hz)}    \\
   \cmidrule{2-3}
   & Nominal Value & Uncertainty \\
   \midrule
   10-bit & 47.264 & 0.40680 \\
   5-bit  & 47.370 & 0.38844 \\
   \bottomrule
   \end{tabular}
\end{table}

\noindent The equation relating frequency and period is shown below. Sample calculation for the 10-bit ADC is shown below.
\[
\begin{aligned}
   f &= \frac{1}{T} \\
       &= \frac{1}{\qty{21158}{\micro\second}} \\
       &= \boxed{\qty{47.264}{\hertz}}
\end{aligned}
\]

\noindent Utilizing error propagation, the random uncertainty in frequency is:
\[
\begin{aligned}
   P_{x'} &= \sqrt{\left(\frac{\partial f}{\partial T} P_x\right)^2} \\
         &= \abs{\frac{\partial f}{\partial T} P_x} \\
         &= \abs{-\frac{1}{T^2} P_x} \\
         &= \frac{P_x}{T^2}\\
         &= \frac{\qty{182.10}{\micro\second}}{\qty{21158}{\micro\second}^2}\\
         &= \boxed{\qty{0.40680}{\hertz}}
\end{aligned}
\]

\subsection{Time-Varying Voltage Peak to Peak Voltage Calculations}
\noindent Below are the resulting peaks and troughs for the 10-bit and 5-bit ADCs for 5 cycles. 
\begin{table}
   \centering
   \caption{Time-Varying Peak and Trough Voltage Measurements for 10-bit and 5-bit ADCs}
   \label{tab:time-varying-peak-to-peak-voltage-measurements} 
   \begin{tabular}{lcccccc}
   \toprule
      & \multicolumn{5}{c}{Peak and Trough Voltage (V)}    \\
   \cmidrule{2-6}
      & Cycle 1    & Cycle 2   & Cycle 3   & Cycle 4   & Cycle 5    \\
   \midrule
   10-bit Trough & 1.212 & 1.234 & 1.234 & 1.206 & 1.199  \\
   5-bit Trough  & 1.225 & 1.237 & 1.228 & 1.192 & 1.215  \\
   10-bit Peak   & 1.763 & 1.772 & 1.763 & 1.743 & 1.766  \\
   5-bit Peak    & 1.76  & 1.772 & 1.766 & 1.747 & 1.76   \\
   \bottomrule
   \end{tabular}
\end{table}


\noindent \textbf{The nominal values for troughs and peaks} were calculated using the \texttt{=AVERAGE()} function in Excel, the standard deviation 
was calculated using the \texttt{=STDEV.S()} function, and the T-distribution inverse was calculated using the \texttt{=T.INV()} function, where $\alpha = 0.05$ 
and $n = 5$. The systematic uncertainty was taken to be the accuracy of the 3.3V reference voltage for 10-bit ADC. and half the resolution for 5-bit ADC.

\begin{table}[h]
   \centering
   \caption{Time-Varying Peak and Trough Voltage Uncertainty Measurements for 10-bit and 5-bit ADCs}
   \label{tab:time-varying-peak-to-peak-voltage-uncertainty-measurements}
   \begin{tabular}{lcccccc}
      \toprule
      & Nominal Value & STDEV & T-Inv & $P_x$ & $B_x$ & $U_x$ \\
      & (V)           & (V)                &       & (V)   & (V)   & (V)   \\
      \midrule
      10-bit Trough & 0.544 & 0.01111 & 2.7764 & 0.01380 & 0.01700 & 0.0219 \\
      5-bit Trough  & 0.542 & 0.008307 & 2.7764 & 0.01031 & 0.05156 & 0.0526 \\
      10-bit Peak   & 1.761 & 0.009274 & 2.7764 & 0.01151 & 0.01700 & 0.0205 \\
      5-bit Peak    & 1.761 & 0.01092 & 2.7764 & 0.01356 & 0.05156 & 0.0533 \\
      \bottomrule
   \end{tabular}
\end{table}
\FloatBarrier
\noindent The total uncertainty, $U_x$, was calculated using the root sum square (RSS) method. An example calculation for the 10-bit trough is shown below:
\[
   \begin{aligned}
      U_x &= \sqrt{P_x^2 + B_x^2} \\
            &= \sqrt{0.01380^2 + 0.01700^2} \\
            &= \boxed{\qty{0.0219}{\volt}}
   \end{aligned}
\]

\noindent Obtaining peak to peak voltage was done by subtracting the trough from the peak. The results are shown in Table \ref{tab:time-varying-peak-to-peak-voltage-results}. 
A sample calculation for 10-bit ADC and equation for peak to peak voltage is shown below:
\[
\begin{aligned}
   V_{\text{p-p}} &= V_{\text{ru}} - V_{\text{rl}} \\
                  &= \qty{1.761}{\volt} - \qty{0.544}{\volt} \\
                  &= \boxed{\qty{1.217}{\volt}}
\end{aligned}
\]

\begin{table}[h]
   \centering
   \caption{Time-Varying Peak to Peak Voltage Results for 10-bit and 5-bit ADCs}
   \label{tab:time-varying-peak-to-peak-voltage-results}
   \begin{tabular}{lcc}
   \toprule
   & Nominal Value & $U_x$ \\
   & (V)           & (V)   \\
   \midrule
   10-bit & 1.217 & 0.0381  \\
   5-bit  & 1.219 & 0.0981   \\
   \bottomrule
   \end{tabular}
\end{table}
\FloatBarrier
\noindent Error propagation was used to calculate the uncertainty in peak to peak voltage. The equation for peak to peak voltage is shown below with 
sample calculations for the 10-bit ADC.
\[
\begin{aligned}
   U_{x'} &= \sqrt{\left(\frac{\partial V_{\text{p-p}}}{\partial V_{\text{ru}}} U_{x, ru}\right)^2 + \left(\frac{\partial V_{\text{p-p}}}{\partial V_{\text{rl}}} U_{x, rl}\right)^2} \\
          &= \sqrt{1^2 0.0219^2 + (-1)^2 0.0205^2} \\
            &= \boxed{\qty{0.0381}{\volt}}
\end{aligned}
\]



% \begin{table}[h]
%     \centering
%     \caption{Time-Varying Frequency Measurements for 10-bit and 5-bit ADCs}
%     \begin{tabular}{lcccccc}
%     \toprule
%        & \multicolumn{5}{c}{Frequency (Hz)}    \\
%     \cmidrule{2-6}
%        & Cycle 1    & Cycle 2   & Cycle 3   & Cycle 4   & Cycle 5    \\
%     \midrule
%     10-bit & 47.619 & 47.295 & 47.331 & 47.357 & 46.729  \\
%     5-bit  & 47.819 & 47.384 & 47.313 & 47.402 & 46.940  \\
%     \bottomrule
%     \end{tabular}
% \end{table}

% \begin{table}[h]
%     \centering
%     \caption{Time-Varying Peak to Peak Voltage Measurements for 10-bit and 5-bit ADCs}
%     \begin{tabular}{lcccccc}
%     \toprule
%        & \multicolumn{5}{c}{Peak to Peak Voltage (V)}    \\
%     \cmidrule{2-6}
%        & Cycle 1    & Cycle 2   & Cycle 3   & Cycle 4   & Cycle 5    \\
%     \midrule
%     10-bit & 1.212 & 1.234 & 1.234 & 1.206 & 1.199  \\
%     5-bit  & 1.225 & 1.237 & 1.228 & 1.192 & 1.215  \\
%     \bottomrule
%     \end{tabular}
% \end{table}

% \begin{table}[h]
%     \centering
%     \label{tab:time-varying-voltage-mean-measurements}
%     \begin{tabular}{lccc}
%            & \multicolumn{3}{c}{Mean Measurements of Time-Varying Voltage} \\
%     \cmidrule{2-4}
%            & Frequency & Peak to Peak & Voltage       \\
%            & (Hz)      & (V)          & (V)           \\
%     \midrule
%     10-bit & 47.266  & 1.217        & 1.135         \\
%     5-bit  & 47.372  & 1.2194       & 1.144         \\
%     Oscilloscope & 48.54 & 1.28 & 1.17 \\
%    \bottomrule
%     \end{tabular}
% \end{table}
% \subsection{Mean Sample Calculation}
% The mean for the 10-bit and 5-bit ADCs was calculated using Excel using the \texttt{=AVERAGE()} function across 5 cycles, which was
% approximately 60 data points. 

% \subsection{Uncertainty Analysis}
% \subsubsection{Frequency Uncertainty}
% \noindent Sample calculations for the 10-bit ADC are shown below. Note that the 5-bit ADC calculations are similar.

% For frequency, standard deviation, $\sigma$, was calculated from Excel using \texttt{=STDEV.S()}.
% For the T-distribution inverse, $t_{\alpha/2, n-1}$, the \texttt{=T.INV()} function was used, where
% $\alpha = 0.05$ and $n = 5$.

% \noindent Using $\sigma = \qty{0.32582}{\hertz}$ and $t_{\alpha/2, n-1} = 2.7764$, the uncertainty in frequency is:
% \[
% \begin{aligned}
%       u(f) &= t_{\alpha/2, n-1} \frac{\sigma}{\sqrt{n}} \\
%             &= 2.7764 \frac{\qty{0.32582}{\hertz}}{\sqrt{5}} \\
%             &= \qty{0.40456}{\hertz}
% \end{aligned}   
% \]

% \subsubsection{Peak to Peak Voltage Uncertainty}
% \noindent For peak to peak voltage, $\sigma = 0.0162$ and $t_{\alpha/2, n-1} = 2.7764$, so the uncertainty in peak to peak voltage is:
% \[
% \begin{aligned}
%       P_x &= t_{\alpha/2, n-1} \frac{\sigma}{\sqrt{n}} \\
%                 &= 2.7764 \frac{\qty{0.0162}{\volt}}{\sqrt{5}} \\
%                   &= \qty{0.0201}{\volt}
% \end{aligned}
% \]

% \noindent For $B_x$, error propagation was used. For a single measurement, the systematic uncertainty was
% $B_{x} = \qty{0.01700}{\volt}$, which was taken from the calibration of the 3.3V reference section. 
% Since the governing equation is:
% \[
% \begin{aligned}
%       V_{\text{p-p}} &= V_{\text{ru}} - V_{\text{rl}}
% \end{aligned}
% \]
% \noindent The propagated systematic uncertainty is:
% \[
%    \begin{aligned}
%       B_{x'} &= \sqrt{\frac{\partial V_{\text{p-p}}}{\partial V_{\text{ru}}}^2 B_{\text{ru}}^2 
%       + \frac{\partial V_{\text{p-p}}}{\partial V_{\text{rl}}}^2 B_{\text{rl}}^2} \\
%             &= \sqrt{1^2 \qty{0.01700}^2 + (-1)^2 \qty{0.01700}^2} \\
%             &= \sqrt{2} (0.01700) \\
%             &= \qty{0.02404}{\volt}
%    \end{aligned}
% \]

% \noindent For total uncertainty of peak to peak voltage, RSS was used:
% \[
% \begin{aligned}
%       U_x &= \sqrt{P_{x}^2 + B_{x}^2} \\
%             &= \sqrt{\qty{0.0201}{\volt}^2 + \qty{0.02404}{\volt}^2} \\
%             &= \qty{0.0313}{\volt}
% \end{aligned}
% \]

% \noindent \textit{Note: For the 5-bit ADC, the accuracy is half the resolution,} $B_{x} = \qty{0.05156}{\volt}$.  

% \begin{table}[h]
%       \centering
%       \caption{Uncertainty Results for Time-Varying Voltage Measurements}
%       \begin{tabular}{lcccc}
%       \toprule
%          & Frequency (Hz) & \multicolumn{3}{c}{Peak to Peak Voltage (V)} \\
%       \cmidrule(lr){2-2} \cmidrule(lr){3-5}
%          & $P_x$ & $P_x$ & $B_x$ & $U_x$ \\
%       \midrule
%       10-bit & 0.4046 & 0.0201 & 0.02404 & 0.0313 \\
%       5-bit  & 0.3886 & 0.0214 & 0.07292 & 0.0760 \\
%       \bottomrule
%       \end{tabular}
% \end{table}
      




