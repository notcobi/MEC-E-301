\FloatBarrier
\section{}
% Assuming the accuracy of the tape measure is 1 mm and the accuracy of the 
% micrometer is 0.003 mm, calculate the total uncertainty in your measurements of b, 
% h and L (use 95% confidence interval for all uncertainty calculations in this lab.)
% stdev	T-INV	P_x	B_x	U_x
% Description	mm		mm	mm	mm
% Distance from centre of strain gauge to weight hanger, L	0.4	2.7764	0.5	1	1
% Width of beam, b	0.001	2.7764	0.001	0.003	0.003
% Height of beam, h	0.002	2.7764	0.002	0.003	0.004

\begin{table}[h]
    \centering
    \caption{Uncertainty in measurements of $b$, $h$, and $L$}
    \label{tab:Q4Uncertainty}
    \begin{tabular}{cccccc}
        \toprule
        Dimension & STDEV & $T_{\text{INV}}$ & $P_x$ & $B_x$ & $U_x$ \\
        & (mm) & & (mm) & (mm) & (mm) \\
        \midrule
        $L$ & 0.4 & 2.7764 & 0.5 & 1 & 1 \\
        $b$ & 0.001 & 2.7764 & 0.001 & 0.003 & 0.003 \\
        $h$ & 0.002 & 2.7764 & 0.002 & 0.003 & 0.004 \\
        \bottomrule
    \end{tabular}
\end{table}

A sample calculation of the uncertainty in $L$ is shown below. The uncertainty in $b$ and $h$ are calculated in a similar manner.

Using Excel's \texttt{STDEV.S} function applied to Table \ref{tab:Q2BeamMeasurements}, \texttt{T.INV} function with $\alpha = 0.05$ and $n = 5$, 
the precision uncertainty $P_x$ is calculated as:
\begin{align*}
    P_x &= \frac{S_x}{\sqrt{n}} \times t_{\alpha/2, n-1} \\
    &= \frac{0.4}{\sqrt{5}} \times 2.7764 \\
    &= \qty{0.5}{mm}
\end{align*}

The bias uncertainty $B_x$ was given to be 1 mm. The total uncertainty $U_x$ is calculated as:
\begin{align*}
    U_x &= \sqrt{P_x^2 + B_x^2} \\
    &= \sqrt{(0.5)^2 + (1)^2} \\
    &= \boxed{\qty{1}{mm}}
\end{align*}



