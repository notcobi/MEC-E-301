\section{}
% Determine the total uncertainty in Δ𝐸0 only for the full bridge when the 1 kg weight 
% was used: As shown in Eq 2, Δ𝐸0
% is the voltage difference measured by the Arduino 
% divided by the gain. Note that 𝐸w and 𝐸nw are both measured by the same device 
% (the Arduino) and thus 𝐸w and 𝐸nw are not independent measurements. This is a 
% special (but common) case in uncertainty analysis. Note that in this case, a 
% systematic bias in the voltage measurement creates no error in the difference 
% between the voltages (i.e., if the Arduino always measured 10 mV too low it would 
% result in no error in the voltage difference). However, other sources of bias
% (resolution or linearity) could still cause errors. In this case, resolution is the largest 
% bias uncertainty and would have a maximum value equal to the resolution of the 
% Arduino (3.3 V/2
% 10). There would also be precision uncertainty in the 
% measurement of the voltage difference (Δ𝐸w). Thus, calculate the total uncertainty 
% in Δ𝐸w then use the propagation of uncertainty to find the total uncertainty in Δ𝐸0
% assuming the uncertainty in the gain is 1%.

\subsection{Uncertainty in $\Delta E_w$}
% stdev	T-INV	P_x	B_x	U_x
% Description	mm		mm	mm	mm
% delta Ew	0.002683282	2.5706	0.0028	0.003222656	0

\begin{table}[h]
    \centering
    \caption{Uncertainty in $\Delta E_w$}
    \label{tab:Q5UncertaintyDeltaEw}
    \begin{tabular}{cccccc}
        \toprule
        Dimension & STDEV & T-INV & $P_x$ & $B_x$ & $U_x$ \\
        & (V) & & (V) & (V) & (V) \\
        \midrule
        $\Delta E_w$ & 2.683E-03 & 2.5706 & 2.8E-03 & 3.223E-03 & 4.3E-03 \\
        \bottomrule
    \end{tabular}
\end{table}

Again, standard deviation is calculated using Excel's \texttt{STDEV.S} function. T-INV is calculated using Excel's \texttt{T.INV} function with $\alpha = 0.05$ and $n = 6$. 
The precision uncertainty $P_x$ is calculated as:
\begin{align*}
    P_x &= \frac{S_x}{\sqrt{n}} \times t_{\alpha/2, n-1} \\
    &= \frac{2.683E-03}{\sqrt{6}} \times 2.5706 \\
    &= \qty{2.8E-03}{V}
\end{align*}

The bias uncertainty $B_x$ is the resolution, 
\begin{align*}
    B_x &= \frac{3.3}{2^{10}} \\
    &= \qty{3.223E-03}{V}
\end{align*}

The total uncertainty $U_x$ is calculated as:
\begin{align*}
    U_x &= \sqrt{P_x^2 + B_x^2} \\
    &= \sqrt{(2.8\times 10^{-3})^2 + (3.223 \times 10^{-3})^2} \\
    &= \boxed{\qty{4.3E-03}{V}}
\end{align*}

\subsection{Uncertainty in $\Delta E_0$}
The equation for $\Delta E_0$ is given by:
\begin{align*}
    \Delta E_0 &= \frac{\Delta E_w}{G} \\
    &= \frac{4.3 \times 10^{-3}}{2.09} \\
    &= \boxed{\qty{2.06E-03}{V}}
\end{align*}

Calculating the partials,
\begin{align*}
    \frac{\partial \Delta E_0}{\partial \Delta E_w} &= \frac{1}{G} \\
    \frac{\partial \Delta E_0}{\partial G} &= -\frac{\Delta E_w}{G^2}
\end{align*}

Calculating the partials times the uncertainty (the nominal value for $\Delta E_w$ is 1.160 V),
\begin{align*}
    \frac{\partial \Delta E_0}{\partial \Delta E_w} \delta \Delta E_w &=  \frac{1}{3000} \times 4.3 \times 10^{-3} \\
    & = 1.4\times 10^{-6} \\
    \frac{\partial \Delta E_0}{\partial G} \delta G &=  -\frac{1.160}{3000^2} \times 1 \times 3000 \\
    & = -3.9\times 10^{-6}
\end{align*}

The total uncertainty is the RSS of the partials times the uncertainty in each variable:
\begin{align*}
    \delta \Delta E_0 &= \sqrt{(1.4\times 10^{-6})^2 + (-3.9\times 10^{-6})^2} \\
    &= \boxed{\qty{3.6E-5}{V}}
\end{align*}